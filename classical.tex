\documentclass{article}
% \usepackage{prooftree}
\usepackage{tipa}
\usepackage{amssymb}
\usepackage{amsmath}
\usepackage{amsthm}
\usepackage{mathrsfs}
\usepackage{tikz-cd}
\usepackage{graphicx}
% \usepackage{prooftrees}
\usepackage{bussproofs}
\usepackage{hyperref}
\usepackage{cleveref}
\usepackage{soul}
\usepackage{stmaryrd}
% \usepackage[a4paper, bindingoffset=0cm, left=1cm, right=1cm top=2cm, bottom=3cm, footskip=1cm]{geometry}

% \usepackage{xcolor} \pagecolor[rgb]{0,0,0} \color[rgb]{1,1,1}

\newcommand{\successor}{\mathsf{succ}}
\newcommand{\N}{\mathbb{N}}
\newcommand{\nat}{\mathsf{nat}}
\newcommand{\recN}{\mathsf{rec}_{\mathbb{N}}}
\newcommand{\bool}{\mathsf{bool}}
\newcommand{\true}{\mathsf{true}}
\newcommand{\false}{\mathsf{false}}
\newcommand{\recbool}{\mathsf{rec_{bool}}}
\newcommand{\proc}[2]{\langle{#1}\mid{#2}\rangle}
\newcommand{\emptystack}{\pi_\mathsf{empty}}
\newcommand{\iif}{\mathsf{if}}
\newcommand{\eelse}{\mathsf{else}}
\newcommand{\tthen}{\mathsf{then}}
\newcommand{\depforall}[1]{\forall^\N #1.}
\newcommand{\pole}{{\bot\mkern-10mu\bot}}
\newcommand{\realizes}{\Vdash}
\newcommand{\oracle}[2]{\lambda\mkern-15mu\lambda\ #1. #2}
\newcommand{\cc}{\mathsf{cc}}
\newcommand{\cont}{\mathsf{k}}
\newcommand{\prop}{\mathsf{prop}}
\newcommand{\recnat}[1]{\mathsf{recnat}_{#1}}
\newcommand{\typeinterp}[1]{{\llbracket #1 \rrbracket}}
\newcommand{\terminterp}[2]{\Vert #1 \Vert_{#2}}
\newcommand{\truthinterp}[2]{\vert #1 \vert_{#2}}
\newcommand{\powerset}[1]{\mathcal{P}(#1)}
\newcommand{\eid}{e_\mathsf{id}}
\newcommand{\etop}{e_\mathsf{top}}
\newcommand{\econj}[2]{<|#1, #2|>}
\newcommand{\efst}{e_\mathsf{fst}}
\newcommand{\esnd}{e_\mathsf{snd}}
\newcommand{\eeval}{e_\mathsf{eval}}
\newcommand{\elambda}[1]{\lambda #1}
\newcommand{\emptyval}{{\rho_\mathsf{empty}}}
\newcommand{\cons}{\mathsf{cons}}
\newcommand{\nil}{\mathsf{nil}}
\newcommand{\reclist}{\mathsf{rec}_\mathsf{list}}
\newcommand{\nth}{\mathsf{nth}}
\newcommand{\length}[1]{{|#1|}}

\newtheorem{definition}{Definition}
\newtheorem{lemma}{Lemma}
\newtheorem{theorem}{Theorem}
\newtheorem{fact}{Fact}
\newtheorem{corollary}{Corollary}
\newtheorem{claim}{Claim}
\newtheorem{property}{Propetry}
\newtheorem{notation}{Notation}

\title{Computational Content of the Classical Axiom of Countable Choice}
\author{Vladimir Ivanov}
\date{\today}

\begin{document}
    
\maketitle

\section{The Computational System}

\begin{definition}[Lambda Terms]
    Let $\mathsf{Var}$ be a countably infinite set of variable names.
    The notions of free variable and closed term are defined as usual.
    We define the set $\Lambda_{open}$ of not necessarily closed lambda terms, and take $\Lambda$ to be the set of closed terms of $\Lambda_{open}$. When we say lambda term, we mean closed lambda term. $\Lambda_{open}$ is defined as follows
    \begin{align*}
        \Lambda_{open} := & \mid x \in \mathsf{Var} \\
        & \mid \Lambda_{open}\ \Lambda_{open} \\
        & \mid \lambda x. \Lambda_{open} & \text{where $x \in \mathsf{Var}$} \\
        & \mid 0 \mid \successor \mid \recN & \text{constructors and the recursor for naturals} \\
        % & \mid \true \mid \false \mid \recbool & \text{constructors and the recursor for $\bool$}\\
        & \mid \cons \mid \nil \mid \reclist & \text{the constructors and recursor for lists} \\
        & \mid \Phi & \text{the bar recursion operator} \\
        & \mid \oracle{n}{t_n} \text{ where $(t_n)_{n \in \N} \subseteq \Lambda$} & \text{when applied to $\successor^n 0$, reduces to $t_n$} \\
        & \mid \cc & \text{call/cc} \\
        & \mid \cont_\pi & \text{continuation, where $\pi$ is a stack}
    \end{align*}
\end{definition}

$\cons$ should be thought of as appending an element at the end of a list and not at the beginning. Note that this implies that the head of a list is a list and the tail of a list is an element, contrary to what's usual.

\begin{definition}[Stacks]
    A stack is a finite list of lambda terms.
    We write $\Pi$ for the set of all stacks.
    We write $t \cdot \pi$ for prepending a lambda term to a stack and $\pi \cdot \pi'$ for concatenating two stacks. We write $\emptystack$ for the empty stack.
\end{definition}

\begin{definition}[Prooflike Term]
    A lambda term is prooflike if it does not contain $\cont{}$ or $\oracle{}{}$
\end{definition}

\begin{notation}
    We write $\overline{n}$ for $\successor^n\ 0$.

    We write $[]$ for $\nil$ and $\ell_\mathsf{head} :: x_\mathsf{tail}$ for $\cons\ \ell_\mathsf{head}\ x_\mathsf{tail}$, the associativity of $\ell :: x_1 :: x_2 :: \dots :: x_n$ is $((\dots((\ell :: x_1) :: x_2) \dots ) :: x_n$.
    We write $[x_1, x_2, \dots, x_n]$ for $\nil :: x_1 :: x_2 :: \dots :: x_n$.
\end{notation}

\begin{notation}
    Let $\vec{t} \subseteq \Lambda$ and $\vec{\pi} \subseteq \Pi$.
    We write $\vec{t} \cdot \vec{\pi}$ for $\{ t \cdot \pi \mid t \in \vec{t}, \pi \in \vec\pi \}$.
    For $t \in \Lambda$ and $\pi \in \Pi$, we write $t \cdot \vec\pi$ for $\{ t \} \cdot \vec\pi$ and $\vec{t} \cdot \pi$ for $\vec{t} \cdot \{ \pi \}$.
\end{notation}

\begin{definition}[Process]
    A process is a pair $\proc{t}{\pi}$ of a lambda term and a stack.
    We write $\Lambda \times \Pi$ for the set of all processes.
\end{definition}

\begin{definition}[Reduction Relation]
    The big step reduction relation $\succ$ is the smallest transitive and reflexive relation which staisfies the following. The term $\nth$ used in the rule for $\Phi$ will be defined right after.
    \begin{align*}
        \proc{t u}{\pi} & \succ \proc{t}{u \cdot \pi} \\
        \proc{\lambda x. t}{u \cdot \pi} & \succ \proc{t[x := u]}{\pi} \\
        \proc{\recN}{t_0 \cdot t_\successor \cdot 0 \cdot \pi} & \succ \proc{t_0}{\pi} \\
        \proc{\recN}{t_0 \cdot t_\successor \cdot \successor\ n \cdot \pi} & \succ \proc{t_\successor\ n\ (\recN\ t_0\ t_\successor\ n)}{\pi} \\
        % \proc{\recbool}{t_\true \cdot t_\false \cdot \true \cdot \pi} & \succ \proc{t_\true}{\pi} \\
        % \proc{\recbool}{t_\true \cdot t_\false \cdot \false \cdot \pi} & \succ \proc{t_\false}{\pi} \\
        \proc{\reclist}{t_\nil \cdot t_\cons \cdot [] \cdot \pi} & \succ \proc{t_\nil}{\pi} \\
        \proc{\reclist}{t_\nil \cdot t_\cons \cdot (\ell_\mathsf{head} :: x_\mathsf{tail}) \cdot \pi} & \succ \proc{t_\cons\ x_\mathsf{tail}\ \ell_\mathsf{head}\ (\reclist\ t_\nil\ t_\cons\ \ell_\mathsf{head})}{\pi} \\
        \proc{\Phi}{H \cdot P \cdot \ell \cdot \pi} & \succ \proc{P\ (\lambda m.\ \nth\ m\ (\ell :: H\ (\lambda z.\ \Phi\ H\ P\ (\ell :: z))))}{\pi} \\
        \proc{\oracle{n}{t_n}}{\overline{n} \cdot \pi} & \succ \proc{t_n}{\pi} \\
        \proc{\cc}{t \cdot \pi} & \succ \proc{t}{\cont_\pi \cdot \pi} \\
        \proc{\cont_{\pi'}}{t \cdot \pi} & \succ \proc{t}{\pi'}
    \end{align*}
\end{definition}

\begin{notation}
    For $t, u \in \Lambda$, we write $t \succ u$ for $\forall \pi \in \Pi. \proc{t}{\pi} \succ \proc{u}{\pi}$.
\end{notation}

\begin{definition}[$\nth$]
    We use $\reclist$ and $\recN$ to define a (prooflike) lambda term $\nth$ such that for all $n \in \N$ and $x_0, \dots, x_{k-1} \in \Lambda$,
    \begin{align*}
        \nth\ \overline{n}\ [x_0, \dots, x_{k-1}] & \succ x_n & \text{if $n < k$} \\
        \nth\ \overline{n}\ [x_0, \dots, x_{k-1}] & \succ x_{k-1} & \text{if $n \ge k$ and $k > 0$}
    \end{align*}
\end{definition}

\begin{definition}[Length of a List]
    For a term $\ell$ of the form $[x_0, \dots, x_{k-1}]$, we define, in the metatheory, $\length{\ell}$ to be $k$.
\end{definition}

\section{Realizability}

\subsection{Logic}

We define higher order logic in this section.

\begin{description}
    \item[types] Types are syntactically defined as $\tau, \sigma, \dots := \nat \mid \prop \mid \tau \rightarrow \sigma$
    \item[variables] For each type $\tau$, take a countably infinite set of variables of this type denoted $x^\tau, y^\tau, \dots$ or $x, y, \dots$, where $x$ and $x^\tau$ is the same variable name.
    \item[Formulas] Formulas are tied with a type and defined inductively as follows
    \begin{description}
        \item[variable] If $x^\tau$ is a variable of type $\tau$, then $x^\tau$ is a formula of type $\tau$.
        \item[abstraction] If $x^\tau$ is a variable of type $\tau$ and $M$ is a formula of type $\sigma$, then $\lambda x^\tau. M$ is a term of type $\tau \rightarrow \sigma$.
        \item[application] If $M$ is a formula of type $\tau \rightarrow \sigma$ and $N$ is a formula of type $\tau$, then $MN$ is a formula of type $\sigma$.
        \item[zero] $0$ is a formula of type $\nat$.
        \item[successor] $\successor$ is a formula of type $\nat \rightarrow \nat$.
        \item[recursor for naturals] For every type $\tau$, $\recnat\tau$ is a formula of type $\tau \rightarrow (\nat \rightarrow \tau \rightarrow \tau) \rightarrow \nat \rightarrow \tau$.
        \item[implication] If $M$ and $N$ are formulas of type $\prop$, then $M \Rightarrow N$ is a formula of type $\prop$.
        \item[universal quantification] If $x^\tau$ is a variable of type $\tau$ and $M$ is a formula of type $\prop$, then $\forall x^\tau. M$ is a formula of type $\prop$.
        \item[dependent universal quantification] If $x^\nat$ is a variable of type $\nat$ and $M$ is a formula of type $\prop$, then $\depforall{x} M$ is a formula of type $\prop$.
        \item[equality] If $M$ and $N$ are formulas of type $\nat$, then $M = N$ is a formula of type $\prop$ (should I define equality for any type?).
        \item[top and bottom] $\top$ and $\bot$ are formulas of type $\prop$.
    \end{description}
\end{description}

As usual, we write $\neg M$ for $M \rightarrow \bot$.

Note that since $\cc$ realizes Peirce's law, we do not need to define existential quantification since it can be replaced by $\neg\forall\neg$.

\subsection{The Realizability Relation}

\begin{definition}[$\pole$]
    We define the pole $\pole \subseteq \Lambda \times \Pi$ to be TO DO.

    Note that $\pole$ is closed by anti-reduction, that is, if $p \succ q$ and $q \in \pole$, then $p \in \pole$.
\end{definition}

\begin{definition}[interpretation of types]
    The interpretation $\typeinterp{\tau}$ of a type $\tau$ is a set defined by induction over the syntax of $\tau$ by
    \begin{align*}
        \typeinterp{\nat} & := \N \\
        \typeinterp{\prop} & := \powerset{\Pi} \\
        \typeinterp{\tau \rightarrow \sigma} & := \typeinterp{\sigma}^\typeinterp{\tau}
    \end{align*}
\end{definition}

\begin{definition}[valuation]
    A valuation $\rho$ is a partial function from the set of variables which to each variable of type $\tau$ associates an element of the set $\typeinterp{\tau}$. We furthermore require that $\rho$ be defined at at most finitely many points.

    For a valuation $\rho$, a variable $x$ of type $\tau$, and $y \in \typeinterp{\tau}$, we write $\rho; x \mapsto y$ for the valuation which maps $x$ to $y$ and every $x' \ne x$ to $\rho(x')$.

    The empty valuation $\emptyval$ is the valuation which is defined nowhere.
\end{definition}

\begin{definition}
    For $\vec\pi \subseteq \Pi$, let
    \[\vec\pi^\pole \subseteq \Lambda = \{ t \in \Lambda \mid \forall \pi \in \vec\pi. \proc{t}{\pi} \in \pole \}\]
\end{definition}

\begin{definition}[intetrpretation of terms]
    Let $\rho$ be a valuation.
    For a term $M$ of type $\tau$ such that $\rho$ is defined at all the free variables of $M$, we define the falsity interpretation $\terminterp{M}{\rho} \in \typeinterp{\tau}$ by syntactic induction over $M$ as follows
    \begin{align*}
        \terminterp{x}{\rho} & := \rho(x) \\
        \terminterp{\lambda x^\tau. M}{\rho} & := v \in \typeinterp{\tau} \mapsto \terminterp{M}{\rho; x \mapsto v} \\
        \terminterp{MN}{\rho} & := \terminterp{M}{\rho}(\terminterp{N}{\rho}) \\
        \terminterp{0}{\rho} & := 0 \\
        \terminterp{\successor}{\rho} & := n \mapsto n + 1 \\
        \terminterp{\recnat{\tau}}{\rho} & := P_0 \mapsto P_\successor \mapsto n \mapsto \begin{cases} P_0 & \text{if $n = 0$} \\ P_{\successor}(n-1)(\terminterp{\recnat{\tau}}{\rho}(P_0)(P_\successor)(n-1)) & \text{otherwise} \end{cases} \\
        \terminterp{M = N}{\rho} & := \begin{cases} \emptyset & \text{if $\terminterp{M}{\rho} = \terminterp{N}{\rho}$ in the standard model of $\N$} \\
        \powerset{\Pi} & \text{otherwise} \end{cases} \\
        \terminterp{M \Rightarrow N}{\rho} & := \terminterp{M}{\rho}^\pole \cdot \terminterp{N}{\rho} \\
        \terminterp{\forall x^\tau. M}{\rho} & := \bigcup_{v \in \typeinterp{\tau}} \terminterp{M}{\rho, x \mapsto v} \\
        \terminterp{\depforall{x} M}{\rho} & := \bigcup_{n \in \N} \overline{n} \cdot \terminterp{M}{\rho; x \mapsto \overline{n}} \\
        \terminterp{\top}{\rho} & := \emptyset \\
        \terminterp{\bot}{\rho} & := \powerset{\Pi} \\
    \end{align*}
\end{definition}

\begin{notation}
    Let $M$ be a formula, $t$ be a term, and $\vec\pi \subset \Pi$.
    We write $\truthinterp{M}{\rho}$ for $\terminterp{M}{\rho}^\pole$, $t \realizes_\rho M$ for $t \in \truthinterp{M}{\rho}$, and $t \realizes \vec\pi$ for $t \in \vec\pi^\pole$.

    We write $\terminterp{\cdot}{}$ for $\terminterp{\cdot}{\emptyval}$, $\truthinterp{\cdot}{}$ for $\truthinterp{\cdot}{\emptyval}$, and $\realizes$ for $\realizes_\emptyval$.
\end{notation}

\subsection{Properties}

\begin{property}\label{antired}
    If $t \succ u$ and $u \realizes_\rho M$, then $t \realizes_\rho M$.
\end{property}

This property will be used a lot without being explicitly mentioned.

\begin{proof}
    Suppose $t \succ u$ and $u \realizes_\rho M$, that is, $\forall \pi \in \terminterp{M}{\rho}. \proc{u}{\pi} \in \pole$.
    Let $\pi \in \terminterp{M}{\rho}$, we need to show that $\proc{t}{\pi} \in \pole$.
    This is true because $\proc{t}{\pi} \succ \proc{u}{\pi}$ and the pole is closed by anti-reduction.
\end{proof}

\begin{property}\label{implieselim}
    If $t \realizes_\rho M \Rightarrow N$ and $u \realizes_\rho M$ then $t u \realizes_\rho N$.
\end{property}

\begin{proof}
    Suppose $t \realizes_\rho M \Rightarrow N$ and let $u \in \Lambda$ be such that $u \realizes_\rho M$.
    Let $\pi \in \terminterp{N}{\rho}$.
    We need to show that $\proc{tu}{\pi} \in \pole$.
    But $\proc{tu}{\pi} \succ \proc{t}{u \cdot \pi}$ and $\proc{t}{u \cdot \pi} \in \pole$ since $u \cdot \pi \in \terminterp{M \Rightarrow N}{\rho}$.    
\end{proof}

\begin{property}\label{impliesintro}
    Suppose that for all lambda term $u$, if $u \realizes_\rho M$, then $t[x := u] \realizes_\rho N$.
    Then $\lambda x. t \realizes_\rho M \Rightarrow N$.
\end{property}

\begin{proof}
    Let $\pi \in \terminterp{M \Rightarrow N}{\rho}$, that is, $\pi = u \cdot \pi'$ for some $v \in \truthinterp{M}{\rho}$ and $\pi' \in \terminterp{N}{\rho}$.
    We need to show that $\proc{\lambda x. t}{u \cdot \pi'} \in \pole$, but this reduces to $\proc{t[x := u]}{\pi'}$, which is in the pole by assumption.
\end{proof}

\begin{property}[Consistency]
    There is no $\rho$ and prooflike $t$ such that $t \realizes_\rho \bot$.
\end{property}

\begin{proof}
    TO DO
\end{proof}

\begin{property}\label{reabot}
    If $t \realizes_\rho \bot$ then $t \realizes_\rho M$ for all formula $M$.
\end{property}

\begin{proof}
    Realizing $\bot$ is a universal quantification over a bigger set than realizing $M$.
\end{proof}

\begin{property}\label{depforallelim}
    If $t \realizes_\rho \depforall{x} M$ then for all $n \in \N$, $t \overline{n} \realizes_{\rho; x \mapsto \overline{n}} M$.
\end{property}

\begin{proof}
    Suppose $t \realizes_\rho \depforall{x} M$.
    Let $n \in \N$ and $\pi \in \terminterp{M}{\rho; x \mapsto \overline{n}}$.
    We need to show that $\proc{t \overline{n}}{\pi} \in \pole$.
    It suffices to show that $\proc{t}{\overline{n}\cdot\pi} \in \pole$, which is the case since $\overline{n} \cdot \pi \in \overline{n} \cdot \terminterp{M}{\rho; x \mapsto \overline{n}} \subseteq \terminterp{\depforall{x} M}{\rho}$.
\end{proof}

\begin{property}\label{depforallintro}
    Suppose that for all $n \in \N$, $t[x := \overline{n}] \realizes_{\rho; x \mapsto \overline{n}} M$.
    Then $\lambda x. t \realizes_\rho \depforall{x} M$.
\end{property}

\begin{property}\label{depforallintrooracle}
    Suppose for all $n \in \N$, $t_n \realizes_{\rho, x \mapsto \overline{n}} M$.
    Then $\oracle{n}{t_n} \realizes_\rho \depforall{x} M$.
\end{property}

\begin{proof}
    Let $\pi \in \terminterp{\depforall{x} M}{\rho}$, that is, $\pi = \overline{n} \cdot \pi'$ for some $n \in \N$ and $\pi' \in \terminterp{M}{\rho; x \mapsto \overline{n}}$.
    We have to show that $\proc{\oracle{n}{t_n}}{\overline{n} \cdot \pi'} \in \pole$.
    But this process reduces to $\proc{t_n}{\pi'}$ which is in the pole by hypothesis.
\end{proof}

\begin{proof}
    Let $\pi \in \terminterp{\depforall{x} M}{\rho}$, that is, $\pi = \overline{n} \cdot \pi'$ for some $n \in \N$ and $\pi' \in \terminterp{M}{\rho; x \mapsto \overline{n}}$.
    We need to show $\proc{\lambda x. t}{\overline{n} \cdot \pi'} \in \pole$.
    But this process reduces to $\proc{t[x := \overline{n}]}{\pi'}$, which is in the pole by assumption.
\end{proof}

\begin{property}[Continuity]
    Let $t$ be a term with one free variable $x$.
    Let $M$ be a formula.
    Let $u_0, u_1, \dots \in \Lambda$.
    Suppose that $t[x := \oracle{n}{u_n}] \realizes_\rho M$.
    Then there exists $N \in \N$ such that for all $f \in \Lambda$ such that $\forall n < N.\ f\ \overline{n} \succ u_n$ we have $t[x := f] \realizes_\rho M$.
\end{property}

\begin{proof}
    TO DO
\end{proof}

\begin{property}\label{typeinterpnonempty}
    For all type $\tau$, the set $\typeinterp{\tau}$ is nonempty.
\end{property}

\begin{proof}
    An element of $\typeinterp{\tau}$ can be defined by induction over the syntax of $\tau$ as follows:
    \begin{align*}
        e(\nat) & := 0 \\
        e(\prop) & := \emptyset \\
        e(\tau \rightarrow \sigma) & := x \mapsto e(\sigma)
    \end{align*}
\end{proof}

\section{The Induced Evidenced Frame}

\begin{definition}
    Define an evidenced frame by taking
    \begin{description}
        \item[propositions] $\Phi$ is $\powerset{\Pi}$
        \item[evidence] $E$ is the set of prooflike lambda terms.
        \item[evidence relation] For $\phi_1, \phi_2 \in \Phi$ and $e \in E$, $\phi_1 \xrightarrow{e} \phi_2$ if and only if $e \realizes \phi_1^\pole \cdot \phi_2$.
        \item[top] $\top = \emptyset$.
        \item[conjunction] For $\phi_1, \phi_2 \in \Phi$, $\phi_1 \wedge \phi_2$ is $\bigcup_{\phi \in \Phi} (\phi_1^\pole \cdot \phi_2^\pole \cdot \phi)^\pole \cdot \phi$.
        \item[universal implication] For $\phi_1 \in \Phi$ and $\vec\phi \subseteq \Phi$, $\phi_1 \supset \vec\phi$ is $\phi_1^\pole \cdot \bigcup \vec\phi$.
    \end{description}
     
    We now prove that it satisfies all the axioms of an evidenced frame.
\end{definition}

\begin{proof}
    We will be using the fact that if for all term $u$ such that $u \realizes \phi_1$ we have $t[x := u] \realizes \phi_2$, then $\phi_1 \xrightarrow{\lambda x. t} \phi_2$.
    This can be proved the same way as \cref{impliesintro}.
    \
    \begin{description}
        \item[reflexivity] Take $\eid = \lambda x. x$. If $t \realizes \phi$ then $x[x := t] = t \realizes \phi$.
        
        \item[top] Take $\etop = \lambda x. x$. If $t \realizes F$ then $x[x := t] = t \realizes \top$ since every lambda term realizes $\top$, which can be seen by unfolding the definitions of $\top$ and $\realizes$.
        
        \item[conjunction elimination] Take $\efst = \lambda t. t (\lambda x. \lambda y. x)$.
        $\lambda x. \lambda y. x \realizes \phi_1^\pole \cdot \phi_2^\pole \cdot \phi_1$ since if $t \realizes \phi_1$ and $u \realizes \phi_2$ then $x[x := t, y := u] = t \realizes \phi_1$.
        Thus if $t \realizes \phi_1 \wedge \phi_2$ we have $t \realizes (\phi_1^\pole \cdot \phi_2^\pole \cdot \phi_1)^\pole \cdot \phi_1$ so $t (\lambda x. \lambda y. x) \realizes \phi_1$, which concludes.
        We do the same thing for $\esnd$.
        
        \item[conjunction introduction] Take $\econj{e_1}{e_2} = \lambda t. \lambda u. u (e_1 t) (e_2 t)$.
        Suppose $e_1 \realizes \phi^\pole \cdot \phi_1$, $e_2 \realizes \phi^\pole \cdot \phi_2$, and $t \realizes \phi$.
        We need to show $\lambda u. u (e_1 t) (e_2 t) \realizes \phi_1 \wedge \phi_2$.
        Let $\phi' \in \Phi$ and $u \realizes \phi_1^\pole \cdot \phi_2^\pole \cdot \phi'$.
        We need to show $u (e_1 t) (e_2 t) \realizes \phi'$, which is the case since $e_1 t \realizes \phi_1$ and $e_2 t \realizes \phi_2$.
        
        \item[universal implication introduction] Take $\lambda{e} = \lambda t. \lambda u. e (\lambda v. v t u)$.
        Let $\phi_1, \phi_2 \in \Phi$, $\vec\phi \subseteq \Phi$, and $e \in E$ such that $\forall \phi \in \vec\phi. \phi_1 \wedge \phi_2 \xrightarrow{e} \phi$.
        We need to show that for all $t \realizes \phi_1$, $\lambda u. e (\lambda v. v t u) \realizes \phi_1 \supset \vec\phi$, that is, for all $u \realizes \phi_2$ and $\phi \in \vec\phi$, $e (\lambda v. v t u) \realizes \phi$, which is true since $e \realizes (\phi_1 \wedge \phi_2)^\pole \cdot \phi$ and $\lambda v. v t u \realizes \phi_1 \wedge \phi_2$, as seen for conjunction introduction.
        
        \item[universal quantification elimination] Take $\eeval = \lambda t. (t (\lambda x. \lambda y. x)) (t (\lambda x. \lambda y. y))$.
        Suppose $t \realizes (\phi_1 \supset \vec\phi) \wedge \phi_1$ and let $\phi \in \vec\phi$, we need to show $(t (\lambda x. \lambda y. x)) (t (\lambda x. \lambda y. y)) \realizes \phi$, which is the case since $t (\lambda x. \lambda y. x) \realizes \phi_1^\pole \phi$ and $t (\lambda x. \lambda y. y) \realizes \phi_1$, as seen for conjunction elimination.
    \end{description}
\end{proof}

\section{Realizability of Countable Choice}

Throughout this section, let $\tau$ be a type.
\begin{definition}[Axiom of Countable Choice]
    We define
    \[AC_{\N, \tau} := \forall R^{\nat \rightarrow \tau \rightarrow \prop}. (\forall n^\nat. \neg \forall i^\tau. \neg R(n, i)) \rightarrow \neg \forall f^{\nat \rightarrow \tau}. \neg \depforall{n} R(n, f(n))\]
\end{definition}

\begin{theorem} We have
    \[ \lambda H.\ \lambda P.\ \Phi\ H\ P\ [] \realizes AC_{\N, \tau} \]
\end{theorem}

\begin{proof}
Let $R_0 \in (\powerset{\Pi}^\typeinterp{\tau})^\N$. Let $\rho = (\emptyval; R \mapsto R_0)$.

Let
\begin{align*}
    H & \realizes_\rho \forall n. \neg \forall i. \neg R(n, i) \\
    P & \realizes_\rho \forall f. \neg \depforall{n} R(n, f(n))
\end{align*}

By \cref{impliesintro}, it suffices to show that
\[\Phi\ H\ P\ [] \realizes_\rho \bot \]

\begin{definition}[Cache]
    $[r_0, r_1, \dots, r_n]$ is a cache if for all $k \le n$, there exists some $i_k \in \typeinterp{\tau}$ such that $r_k \realizes_{\rho; i \mapsto i_k} R(\overline{k}, i)$.
\end{definition}

\begin{lemma}\label{growcache}
    Let $\ell$ be a cache.
    Suppose that $\Phi\ H\ P\ \ell \not\realizes_\rho \bot$.
    Then, there exists $i_\length{\ell} \in \typeinterp{\tau}$ and $r_\length{\ell} \in \Lambda$ such that $\ell :: r_\length{\ell}$ is a cache and $\Phi\ H\ P\ (\ell :: r_\length{\ell}) \not\realizes_\rho \bot$.
\end{lemma}

\begin{proof}
    Since $\ell$ is a cache, for all $k < \length{\ell}$, there exists an $i_k \in \typeinterp{\tau}$ such that $\nth\ \overline{k}\ \ell \realizes_{\rho; i \mapsto i_k} R(\overline{k}, i)$. Define $f_0(k)$ to be such an $i$ for $k < \length{\ell}$ and an arbitrary element of $\typeinterp{\tau}$ for $k \ge \length{\ell}$ (this set is nonempty by \cref{typeinterpnonempty}).

    \ul{There exist $r_\length{\ell} \in \Lambda, i_\length{\ell} \in \typeinterp{\tau}$ such that $r_\length{\ell} \realizes_{\rho; i \mapsto i_\length{\ell}} R(\overline{\length\ell}, i)$.}
    Indeed, if none did exists\footnote{We use excluded middle in the metatheory here.}, then for all $i$, any term would realize $\neg R(\overline{\length\ell}, i)$, thus, any term would realize $\forall i. \neg R(\overline{\length{\ell}}, i)$.
    Thus, $H$ applied to any term would realize $\bot$, namely,
    \[ H\ (\lambda z.\ H\ P\ (l :: z)) \realizes_\rho \bot \].
    Thus
    \[\lambda m. \nth\ m\ (\ell :: H\ (\lambda z.\ H\ P\ (l :: z))) \realizes_\rho \depforall{n} R(n, f_0(n))\]
    In effect, the body applied to $\overline{k}$ for $k < \length{\ell}$ reduces to the $k^\text{th}$ element of the cache and thus realizes $R(k, f_0(k))$ and the body applied to $\overline{k}$ for $k \ge \length{\ell}$ reduces to $H\ (\lambda z.\ H\ P\ (l :: z))$, which realizes $\bot$ and therefore by \cref{reabot} any realizes formula.
    Thus,
    \[P\ (\lambda m. \nth\ m\ (\ell :: H\ (\lambda z.\ H\ P\ (l :: z)))) \realizes_\rho \bot\]
    but $\Phi\ H\ P\ \ell$ reduces to this term and was supposed to not realize $\bot$, which is a contradiction.

    \ul{Now, suppose that for all $r_\length{\ell} \in \Lambda, i_\length{\ell} \in \typeinterp{\tau}$ such that $r_\length{\ell} \realizes_{\rho; i \mapsto i_\length{\ell}} R(\overline{\length\ell}, i)$},
    \[ \Phi\ H\ P\ (\ell :: r_\length{\ell}) \realizes_\rho \bot \]
    \ul{it suffices to find a contradiction.}
    Then by \cref{implieselim} and \cref{reabot}, for all $i_\length{\ell} \in \typeinterp{\tau}$,
    \[ \lambda z.\ \Phi\ H\ P\ (\ell :: z) \realizes_{\rho; i \mapsto i_\length{\ell}} \neg R(\overline{k}, i) \]
    and so by the hypothesis on $H$,
    \[ H\ (\lambda z.\ \Phi\ H\ P\ (\ell :: z)) \realizes_\rho \bot \]
    which has already been shown to lead to a contradiction.
\end{proof}

Now, suppose
\[ \Phi\ H\ P\ [] \not\realizes_\rho \bot \]
it suffices to find a contradiction.

By applying \cref{growcache} repeatedly\footnote{We use the axiom of dependent choice in the metatheory here.}, we get a sequence $r_0, r_1, \dots$ such that for all $k$ there exists $i_k \in \typeinterp{i}$ such that $r_k \realizes_{i \mapsto i_k} R(\overline{k}, i)$.
For each $k$, take $f_0(k)$ to be such an $i_k$.

Then, for all $n$, $r_n \realizes_\rho R(\overline{n}, f_0(\overline{n}))$.
Thus, by \cref{depforallintrooracle}, $\oracle{n}{r_n} \realizes_\rho \depforall{n} R(n, f_0(n))$ and so \[P\ (\oracle{n}{r_n}) \realizes_\rho \bot\]

But then, by continuity, there exists $K$ such that for all $f$, if $\forall k < K.\ f\ \overline{k} \succ r_k$ then $P\ f \realizes_\rho \bot$.
Now, take $\ell = [r_0, \dots, r_{K-1}]$ and $f = \lambda m. \nth\ m\ (\ell :: H\ (\lambda z. \ \Phi\ H\ P\ (\ell :: z)))$.
On the one hand, $\Phi\ H\ P\ \ell \succ P\ f$ and $\phi\ H\ P\ \ell \not\realizes_\rho \bot$ by construction, so $P\ f \not\realizes_\rho \bot$.
On the other hard, for $k < K$, $f\ \overline{k} \succ r_k$, thus $P\ f \realizes_\rho \bot$.
We thus obtain the sought contradiction.

\end{proof}

\end{document}









