\documentclass{article}
% \usepackage{prooftree}
\usepackage{tipa}
\usepackage{amssymb}
\usepackage{amsmath}
\usepackage{amsthm}
\usepackage{mathrsfs}
\usepackage{tikz-cd}
\usepackage{graphicx}
% \usepackage{prooftrees}
\usepackage{bussproofs}
\usepackage{hyperref}
\usepackage{cleveref}

\usepackage[nomargin,inline, draft]{fixme}
% fixme package settings
\fxusetheme{color}
\fxuseenvlayout{color}
\FXRegisterAuthor{em}{aem}{\color{purple}[\'E]}
\FXRegisterAuthor{vi}{avi}{\color{teal}[V]}
% \usepackage[a4paper, bindingoffset=0cm, left=1cm, right=1cm top=2cm, bottom=3cm, footskip=1cm]{geometry}

% \usepackage{xcolor} \pagecolor[rgb]{0,0,0} \color[rgb]{1,1,1}

\newtheorem{definition}{Definition}
\newtheorem{lemma}{Lemma}
\newtheorem{theorem}{Theorem}
\newtheorem{fact}{Fact}
\newtheorem{corollary}{Corollary}
\newtheorem{claim}{Claim}

%\newcommand{\memo}{\mathsf{memo}}
%\newcommand{\lookup}{\mathsf{lookup}}
% \newcommand{\fail}{\mathsf{fail}}
% \newcommand{\introexists}{\mathsf{intro}_\exists}
\DeclareMathOperator{\makeaxiom}{\mathsf{makeaxiom}}
\DeclareMathOperator{\axiom}{\mathsf{axiom}}
\DeclareMathOperator{\fail}{\mathsf{fail}}
\DeclareMathOperator{\assoc}{\mathsf{assoc}}
\DeclareMathOperator{\fix}{\mathsf{fix}}
\DeclareMathOperator{\mem}{\mathsf{mem}}
\DeclareMathOperator{\iif}{\mathsf{if}\ }
\newcommand{\tthen}{\mathbin{\ \mathsf{then}}\ }
\newcommand{\eelse}{\mathbin{\ \mathsf{else}}\ }
\DeclareMathOperator{\introexists}{\mathsf{intro}_\exists}
\DeclareMathOperator{\introforall}{\mathsf{intro}_\forall}
\DeclareMathOperator{\elimexists}{\mathsf{elim}_\exists}
\DeclareMathOperator{\elimforall}{\mathsf{elim}_\forall}
\newcommand{\nat}{\mathsf{nat}}

\newcommand{\N}{\mathbb{N}}
\newcommand{\Sigmapre}{\Sigma_{\mathrm{pre}}}
\newcommand{\Sigmainf}{\Sigma_\infty}
\newcommand{\Sigmapreinf}{\Sigma_{\mathrm{pre}, \infty}}
\newcommand{\septop}{\mathcal{S}_\top}

\title{Computational Content of the Axiom of Choice in Evidenced Frames}
\author{Vladimir Ivanov}
\date{\today}

\begin{document}
    
\maketitle

\section{The Computational System}

The additional codes are $\makeaxiom$, $\axiom_n$ for $n \in \N$ and $c_f$ for $f : \N \rightarrow C$.
The additional reduction relations are
\begin{prooftree}
    \AxiomC{}\UnaryInfC{$\makeaxiom \cdot \overline{k} \downarrow \axiom_k$}
\end{prooftree}
\begin{prooftree}
    \AxiomC{}\UnaryInfC{$c_f \cdot \overline{n} \downarrow f(n)$}
\end{prooftree}
The additional termination rules are
\begin{prooftree}
    \AxiomC{}\UnaryInfC{$\makeaxiom \cdot \overline{k} \downarrow$}
\end{prooftree}
\begin{prooftree}
    \AxiomC{}\UnaryInfC{$c_f \cdot \overline{n} \downarrow$}
\end{prooftree}

\section{The Evidenced Frame}

We take the separator $\septop$. We interpret non determinism angelically\emnote{is there non-deterministic computations?}.


For all $n \in \N$, let $\nat_n$ be the predicate realized only by $\overline{n}$.
We define universal quantification over naturals:
\[\forall n. \phi_n\]
as $\prod \{\nat_n \rightarrow \phi_n | n \in \N\}$
\emnote{$\prod \{\nat_n \supset \phi_n | n \in \N\}$}

We define $\bot$ as the predicate realized by all $\axiom_k, k \in \N$.

We allow arbitrary expressions, and not just pairs of codes, on the left of $\downdownarrows$\emnote{You mean applications of en expression to an expression?}.

\section{Pragmatics}

We write $\lambda x_0 \dots x_k. e[x_0]\dots[x_k]$ for $c_{\lambda^k. e}$.
Not that not all lambda terms can be easily encoded, for example, we can't encode $\lambda x. x (\lambda y. xy)$\emnote{why? if you take: 
$e_2\triangleq 1\cdot 0$, $e_1 \triangleq \lambda.e_2$ and $e_0 \triangleq 0\cdot e_1$, you can encode it with $c_{\lambda.e_0}$, no?}
and should work around by encoding it as $\lambda x. x ((\lambda x', y. x' y) x)$.
We leave such conversions implicit.
We do not always write the $\cdot$.

We write $\exists x \in X. \phi_x$ for $\coprod \{\phi_x \mid x \in X\}$ and omit the $\in X$ when $X$ is clear from context.

\section{Preliminaries}

\begin{fact}
    There exist codes $\introexists, \introforall, \elimexists, \elimforall \in \septop$ such that for all family of proposition $(\phi_n)_{n \in \N}$, all set $X$, and all family of propositions $(\psi_x)_{x \in X}$
    \begin{enumerate}
        \item If for all $n$, $c \cdot \overline{n} \downdownarrows \phi_n$, then $c \xrightarrow{\introforall} \forall n. \phi_n$. \emnote{$c$ is not a formula of the language, what do you mean?}
        \item If $c \models \forall n. \phi_n$ \emnote{what is the definition of $c\models \phi$?}
        , then for all $n$, $\elimforall \cdot c \cdot n \models \phi_n$.
        \item For all $x$, $\psi_x \xrightarrow{\introexists} \exists x. \phi_x$.\emnote{what is the relation between $\psi$ and $\phi$?}
        \item If $\prod \{ \phi_x \mid x \in X \}$ has a realizer, then there exists $x \in X$ such that $\prod \{ \phi_x \mid x \in X \} \xrightarrow{\elimexists} \psi_x$
    \end{enumerate}
\end{fact}
\begin{proof}
    Take
    \[\introforall := \lambda c. \lambda (e_{\mathsf{snd}}; c)\]
    \[\elimforall := \lambda c, n. c\ n\ n\]
    \[\introexists := \lambda c. \lambda(<|c, <|e_{\mathsf{id}}, e_\top|>; e_{\mathsf{eval}}|>; e_{\mathsf{eval}})\]
    \[\elimexists := <|\lambda (e_{\mathsf{fst}}; e_\top; \lambda(e_{\mathsf{fst}; e_{\mathsf{id}}})), |>; e_{\mathsf{eval}}\]
    \emnote{what does the notation $<|c,|>$ stand for?}    
\end{proof}

\begin{definition}[Behaves Like]
    We say that code $c_f'$ behaves like code $c_f$ if $\forall c_a, c_r. c_f \cdot c_a \downarrow c_r \Rightarrow c_f' \cdot c_a \downarrow c_r$.
    If we have an equivalence instead of an implication, we say that they are extensionally equal.
    \emnote{I guess you could define this as an ordering relation $ c'\preccurlyeq c$ on codes, and indeed the induced equality would be the extensional equality on codes.}
\end{definition}

\begin{lemma}\label{substaxiomcongr}
    If $c \cdot \overline{n} \downdownarrows \bot$ for all $k$ and $e \downdownarrows \bot$, then $e[\makeaxiom := c] \downdownarrows \bot$.
\end{lemma}
\begin{proof}
    Structural induction over the proof of $e \downarrow \axiom_k$.
\end{proof}

\begin{lemma}\label{extcongr}
    If $c$ and $c'$ are extensionally equal, so are $e[x := c]$ and $e[x := c']$ for all expression $e$ with a hole $x$.
\end{lemma}
\begin{proof}
    Structural induction over the proof of $e[x := c] \downarrow c_r$.
\end{proof}

\section{The Proof}

Let

\[ KCC' := (\forall n. \nabla \exists i. \neg R(n, i)) \Rightarrow \nabla \exists f. \forall n. \neg R(n, f(n)) \]
\[ \fix_f := (\lambda x. f (x x)) (\lambda x. f (x x)) \]
\[ \fix := \lambda f. \fix_f \]
\[ \Phi_{H, P, \phi, L} := P (\introexists (\introforall (\lambda n. \assoc L\ n\ (\lambda \_. (\lambda m. H m (\lambda q. \phi\ ((m, q):L))) n)))) \]
\[ \Phi_{H, P} := \fix_{\lambda \phi L. \Phi_{H, P, \phi, L}} \]
\[H \models \forall n. \nabla \exists i. \neg R(n, i)\]
\[P \models \neg \exists f. \forall n. \neg R(n, f(n))\]

\emnote{what is $\assoc$?}

It suffices to prove the following lemma, the rest of the proof is stricly identical to the original paper.

\begin{definition}[Cache]
    A cache is a church encoded list $L$ of church encoded pairs of the form $(\overline{n}, q), n \in \N, q \in C$ such that
    \begin{itemize}
        \item For all $(\overline{n}, q) \in L$ there exists $i$\emnote{what is the type of $i$?} such that $q \models R(n, i)$. \emnote{$q \models \neg R(n, i)$ ?}
        \item The first elements of pairs in $L$ are pairwise distinct.
    \end{itemize}
\end{definition}

\begin{lemma}\label{addlist}
    Let $L$ be a chache such that $\Phi_{P, H} \cdot L \not\downdownarrows \bot$. Then there exist $n$ and $q$ such that $(n, q):L$ is a cache and $\Phi_{P, H} \cdot ((n, q):L) \not\downdownarrows \bot$.
\end{lemma}


\begin{proof}
    Define $f_0(n)$ to be $i$ such that $q \models \neg R(n, i)$ if $(n, i)\in L$\emnote{$(n,q)\in L$?}  for some $i$ (\emnote{how do you know that such $i$ exists?}, and take $f_0(n)$ to be arbitrary if $n \notin I$.

    \begin{claim}\label{axiomred}
        We have
        \[ P (\introexists (\introforall (\lambda n. \assoc L\ n\ (\lambda \_. \makeaxiom n)))) \downdownarrows \bot\]
    \end{claim}
    Let $n \in \N$.
    Since $\axiom_n \models \bot$, $\lambda \_. \makeaxiom \overline{n} \models \neg R(n, i)$ for all $i$.
    Thus, \[ \assoc L\ \overline{n}\ (\lambda \_. \makeaxiom \overline{n}) \models \neg R(n, f(n)) \], by the previous discussion if $n$ is not in $L$ and by the definition of a cache if $n$ is in $L$.
    This holds for all $n$, so \[ \introforall (\lambda n. \assoc L\ n\ (\lambda \_. \makeaxiom n)) \models \forall n. \neg R(n, f_0(n)) \].
    Therefore, \[ \introexists (\introforall (\lambda n. \assoc L\ n\ (\lambda \_. \makeaxiom n))) \models \exists f. \forall n. \neg R(n, f(n)) \].
    This concludes since $P \models \neg \exists f. \forall n. \neg R(n, f(n))$.

    \begin{claim}\label{notred}
        $\Phi_{H, P, \Phi_{H, P}, L} \not\downdownarrows \bot$, that is (unfolding $\Phi$),
    \[ P (\introexists (\introforall (\lambda n. \assoc L\ n\ (\lambda \_. (\lambda m. H\ m\ (\lambda q. \Phi_{H, P}\ ((m, q):L)) n))))) \not\downdownarrows \bot\]
    \end{claim}
    \begin{proof}
        Suppose for the sake of contradiction $\Phi_{H, P, \Phi_{H, P}, L} \downdownarrows \bot$, that is, $\Phi_{H, P, \Phi_{H, P}, L} \downarrow \axiom_k$ for some $k$.
        So $(\lambda \phi L'. \Phi_{H, P, \phi L'}) \cdot \Phi_{H, P} \cdot L \downarrow \axiom_k$, and since $\Phi_{H, P} = \fix_{\lambda \phi L'. \Phi_{H, P, \phi, L'}}$, we have $\Phi_{H, P} \cdot L \downarrow \axiom_k$, which was supposed to not hold.
    \end{proof}

    \begin{claim}\label{new}
        There exists $k_{new} \in \N$ such that
        \[ \elimforall H\ \overline{k_{new}}\ (\lambda q. \Phi_{H, P} ((\overline{k_{new}}, q):L)) \not\downdownarrows \bot \]
    \end{claim}
    \begin{proof}
        Suppose that $ \elimforall H\ \overline{k}\ (\lambda q. \Phi_{H, P} ((\overline{k}, q):L)) \downarrow \axiom_l$ for some $l$ for all $k$.
        It now suffices to find a contradiction.
        We have $c \cdot \overline{k} \downarrow \axiom_l$ for some $l$ for all $k$, where $c := \lambda m. \elimforall H\ m\ (\lambda q. \Phi_{H, P} ((m, q):L))$.
        Thus, $c$ behaves like $\makeaxiom$, so by applying \cref{substaxiomcongr} to \cref{axiomred} contradicts \cref{notred}.
    \end{proof}

    Now, let $k$ be as in \cref{new}.
    
    \begin{claim}\label{newrealizer}
        There exists a $q \in C$ such that $q \models \neg R(k, i)$ for some $i$.
    \end{claim}
    \begin{proof}
        Suppose, for the sake of contradiction, that there is no such $q$.
        Since $H \models \forall k. \nabla \exists i. \neg R(n, i)$, it follows from \cref{new} that $\lambda q. \Phi_{H, P} ((k, q):L)$ does not realize $\neg \exists i. \neg R(k, i)$.
        Thus, there exists at least one realizer of $\exists i. \neg R(k, i)$ (since otherwise any code would realize its negation).
        Now, $\exists i \neg R(k, i) \xrightarrow{\elimexists} \neg R(k, i)$ for some $i$, which concludes that there exists a realizer of $\neg R(k, i)$ for some $i$.
    \end{proof}

    \begin{claim}\label{newdifferent}
        $k$ is not in $L$.
    \end{claim}
    \begin{proof}
        For all $n$, $\assoc L\ \overline{n}\ (\lambda m. \iif \mem m\ L \tthen (\lambda x. x) \eelse \makeaxiom m)$ realizes $\neg R(n, f_0(n))$, by the definition of a cache if $n$ is in $L$ and since it the else branch is taken ortherwise.
        Thus, \[P (\introexists (\introforall (\lambda n. \assoc L\ n\ (\lambda m. \iif \mem m\ L \tthen (\lambda x. x) \eelse \makeaxiom n)))) \downarrow \axiom\]
        since it realizes $\bot$.
        Then, since for all $n$, $\assoc L\ n\ \makeaxiom \models \bot$, we have \[P (\introexists (\introforall (\lambda n. \assoc L\ n\ (\lambda m. \makeaxiom)))) \downarrow \axiom\]
        Then, by \cref{substaxiomcongr} and \cref{extcongr},
        \[\iif (\mem \overline{k}\ L) \tthen (\lambda x. x) \eelse \makeaxiom \overline{k} \downdownarrows \bot\]
        Thus, $k$ is not in $L$ since $\lambda x. x \not\models \bot$.
    \end{proof}

    We are finished by taking $n = k_{new}$ from \cref{new} and $q$ from \cref{newrealizer}.
\end{proof}

\end{document}











